\documentclass[a6paper]{article}
\usepackage[margin=5mm]{geometry}
\usepackage{tgheros}
\usepackage[T1]{fontenc}
\renewcommand*\familydefault{\sfdefault}


% \documentclass[a4paper]{article}
\usepackage[utf8]{inputenc}
\usepackage[spanish]{babel}

\author{Albert Ribes}
\title{Conexión a Internet en los aviones}

\begin{document}
    \maketitle

    \section{Puntos sobre los que hablar}
        \begin{itemize}
            \item Cómo se gestiona el cambio de un satélite a otro, o de una antena a otra? Se corta la conexión?
        \end{itemize}


    \section{Introducción}
        \begin{itemize}
            \item No sé si os habéis preguntado cómo funciona la conexión a Internet en los aviones.
            \item Uno se cree Internet está en el aire, pero realmente todo se sigue transmitiendo a través de cables
            \item Y como se puede comprobar, es complicado conectar un avión a través de cables. Hay que buscar otras alternativas
        \end{itemize}

    \section{Problemática}
        \subsection{Un poco de números}
            \begin{itemize}
                \item Ancho de banda que ofrece un cable actualmente
                \item Longitud máxima que puede tener un cable actualmente
                \item Las frecuencias de radio asignadas a cada una de las tecnologías
                \item La distancia máxima de transmisión que se puede conseguir con cada frecuencia de radio
                \item La cantidad observada de pérdidas a través del aire
                \item Cómo afecta la lluvia, presión atmosférica, temperatura, etc. con la transmisión a través del aire
                \item Cuanto pesa una antena
                \item Cuanto cuesta una antena
                \item Altura a la que suele llegar un avión
                \item Cuantos bits es un video de youtube
                \item Cuál es la velocidad de transmisión típica con conexión normal
                \item Cual es la velocidad de transmisión que haría falta únicamente  para las comunicaciones que necesita el piloto: radio, radar, sonar, etc.
                \item Cual es la velocidad de transmisión de un usuario que solo usa WhatsApp, la de uno que solo navega por internet (sin streaming ni videos) y la de uno que quiere ver videos
            \end{itemize}
        \subsection{Otra subsección}
        \begin{itemize}
            \item La transmisión a través del aire pierde mucho con la distancia. Es muy dependiente de la zona en la que estás
            \item El peso de las antenas es grande
            \item La frecuencia de radio asignada para este tipo de conexiones es muy limitada
            \item Conseguir más ancho de banda es complicado
        \end{itemize}
    \section{Historia}
        \subsection{Los primeros aviones}
            \begin{itemize}
                \item Comunicación por radio
                \item Conexión por GPS
            \end{itemize}
    \section{Soluciones propuestas}



\end{document}
